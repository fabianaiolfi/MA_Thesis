% Options for packages loaded elsewhere
\PassOptionsToPackage{unicode}{hyperref}
\PassOptionsToPackage{hyphens}{url}
%
\documentclass[
]{article}
\usepackage{amsmath,amssymb}
\usepackage{iftex}
\ifPDFTeX
  \usepackage[T1]{fontenc}
  \usepackage[utf8]{inputenc}
  \usepackage{textcomp} % provide euro and other symbols
\else % if luatex or xetex
  \usepackage{unicode-math} % this also loads fontspec
  \defaultfontfeatures{Scale=MatchLowercase}
  \defaultfontfeatures[\rmfamily]{Ligatures=TeX,Scale=1}
\fi
\usepackage{lmodern}
\ifPDFTeX\else
  % xetex/luatex font selection
\fi
% Use upquote if available, for straight quotes in verbatim environments
\IfFileExists{upquote.sty}{\usepackage{upquote}}{}
\IfFileExists{microtype.sty}{% use microtype if available
  \usepackage[]{microtype}
  \UseMicrotypeSet[protrusion]{basicmath} % disable protrusion for tt fonts
}{}
\makeatletter
\@ifundefined{KOMAClassName}{% if non-KOMA class
  \IfFileExists{parskip.sty}{%
    \usepackage{parskip}
  }{% else
    \setlength{\parindent}{0pt}
    \setlength{\parskip}{6pt plus 2pt minus 1pt}}
}{% if KOMA class
  \KOMAoptions{parskip=half}}
\makeatother
\usepackage{xcolor}
\usepackage[margin=1in]{geometry}
\usepackage{graphicx}
\makeatletter
\def\maxwidth{\ifdim\Gin@nat@width>\linewidth\linewidth\else\Gin@nat@width\fi}
\def\maxheight{\ifdim\Gin@nat@height>\textheight\textheight\else\Gin@nat@height\fi}
\makeatother
% Scale images if necessary, so that they will not overflow the page
% margins by default, and it is still possible to overwrite the defaults
% using explicit options in \includegraphics[width, height, ...]{}
\setkeys{Gin}{width=\maxwidth,height=\maxheight,keepaspectratio}
% Set default figure placement to htbp
\makeatletter
\def\fps@figure{htbp}
\makeatother
\setlength{\emergencystretch}{3em} % prevent overfull lines
\providecommand{\tightlist}{%
  \setlength{\itemsep}{0pt}\setlength{\parskip}{0pt}}
\setcounter{secnumdepth}{-\maxdimen} % remove section numbering
\ifLuaTeX
  \usepackage{selnolig}  % disable illegal ligatures
\fi
\IfFileExists{bookmark.sty}{\usepackage{bookmark}}{\usepackage{hyperref}}
\IfFileExists{xurl.sty}{\usepackage{xurl}}{} % add URL line breaks if available
\urlstyle{same}
\hypersetup{
  hidelinks,
  pdfcreator={LaTeX via pandoc}}

\author{}
\date{\vspace{-2.5em}}

\begin{document}

\hypertarget{model-specification}{%
\subsubsection{Model Specification}\label{model-specification}}

In the preliminary analysis, I establish a negative correlation between
emigration and the change in incumbent vote share in Romania. Building
upon this finding, I develop two model specifications to test the
hypotheses that service cuts decrease incumbent support. Using two model
specifications has the advantage of providing a more comprehensive and
robust exploration of how service cuts and emigration may individually
and interctively affect incumbnet vote share.

The aim of the first model specification is to analyse the isolated
effect of service cuts on incumbent vote share. Thus service cuts form
the independent variable in this model specification. To ensure that the
effect of service cuts is isolated, I control for emigration, as
emigration rates vary across NUTS3 regions within Romania and, as
previously established, influence incumbent vote share.

\begin{itemize}
\tightlist
\item
  to further isloate the effect of service cuts and emigration on
  incumbent vote share, i add electoral volatility, gdp and unemployment
  rate as control variables
\end{itemize}

Using this model specification, I set up a two-way fixed effects linear
regression model\footnote{\url{https://theeffectbook.net/ch-FixedEffects.html\#multiple-sets-of-fixed-effects}},
using NUTS3 regions and years as fixed effects. By using fixed effects,
I can account for region- and time-specific factors that may confound
results.

This leads to the Model Specification A:

\begin{equation*}
\begin{aligned}
Incumbent \; Vote \; Share \; Change_{n,y} = \beta_n & + \beta_y & \\
& + \beta_1Emigration \; Rate_{n, y} \\
& + \beta_2Service \; Cuts_{n, y} \\
& + \beta_3Electoral \; Volatility_{n, y} \\
& + \beta_4GDP_{n, y} \\
& + \varepsilon
\end{aligned}
\end{equation*}

The subscript \(n\) symbolises a Romanian NUTS3 region and subscript
\(y\) indicates an election year. \(ε\) denotes the error term.

The second model specification is designed to examine if the impact of
service cuts on incumbent vote share is conditioned by emigration. This
examination focuses on two aspects. First, it determines whether a
significant interaction exists between service cuts and emigration.
Second, it assesses how different levels in emigration modify the effect
of service cuts on incumbent support. By analysing the interaction
between the two variables, the model can reveal if emigration increase
or decrease the effects of service cuts on incumbent support, thus
futher testing the hypotheses.

TO DO: justify this interaction! something along the lines of: - service
cuts can be higher in regions with more emigration - service cuts may be
felt more strongly by the electorate in regions hit hard by emigration;
or there the electorate may feel emigration more strongly, thus
developing more grievanes and thus more anti-incumbent sentiment - ``For
example, it might be hypothesized that in areas with higher emigration
rates, the impact of infrastructure (like schools or hospitals) on
voting pattern changes might be different than in areas with lower
emigration rates.''

This second model specification builds upon the first by adding an
interaction term between emigration and service cuts. The second model
specification is otherwise identical to the first, and will also control
for the same variables as the first model specification.

This leads to the Model Specification B:

\begin{equation*}
\begin{aligned}
Incumbent \; Vote \; Share \; Change_{n, y} = \beta_n & + \beta_y & \\
      & + \beta_1Emigration \; Rate_{n, y} \\
      & + \beta_2Service \; Cuts_{n, y} \\
      & + \beta_2 (Emigration \; Rate_{n, y} \times Service \; Cuts_{n, y}) \\
      & + \beta_3Electoral \; Volatility_{n, y} \\
      & + \beta_4GDP_{n, y} \\
      & + \varepsilon
\end{aligned}
\end{equation*}

In both model specifications, each service cut, i.e.~school closings,
hospital closings, third places closings, will be modelled seperately.

\end{document}
